\chapter{Background}

\section{Findings From NTCIR}
The NTCIR-12 lifelogging latent semantic access task had four participants contribute to the automatic retrieval component and one participant for the interactive retrieval component. The best performing automatic team, LIG-MRIM, used computer vision to classify images and not rely on the visual concepts distributed with the task. The interactive team (LEMoRe) outperformed all other automatic teams, but this was expected as this is generally the case with tasks that have both automatic and interactive components. The other three automatic teams were VTIR, III\&CYUT and QUT (the preliminary work done for this research).

LIG-MRIM used dynamic convolutional neural networks and a multi-class support vector machine (MSVM) in order to classify images. Visual indexing is composed of two parts: three deep convolutional neural network models (AlexNet, GoogleNet and Visual Geometry Grouping (VGG)) process each image. The output is normalised and has principal component analysis (PCA) performed on it. These outputs are then concatenated together. The same normalisation and PCA process is repeated and fed into the MSVM. The output of this is concatenated with the VGG data. The second part involves temporally naming times of the day in order to attempt to extract semanitic meaning from times of the day. While this team submitted runs that fit the definition of automatic for the task, queries were generated manually from the topics by some expert.

III\&CYUT used a traditional textual based approach to lifelog retrieval. A word2vec model for the CAFFE concepts for each image was used to try to add more semantic meaning to each image. Query expansion was also used on every keyword.

The VTIR team attempted to exploit location metadata in the images. This team choose 3000 images randomly and labelled them with a rich semantic location ontology. More concepts were added by applying the WordNet database to find cognitive synonyms.

Finally, the LEMoRe team, which used an interactive approach, combined existing technologies and methodologies in order to develop a search engine. Colour correlogram, edge histogram, joint composite descriptor and pyramid histogram of oriented graphics are used by the image retrieval system as features to retrieve images. Both a novice and an expert used the system to produce runs.