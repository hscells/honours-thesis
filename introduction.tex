\chapter{Introduction}

What if you had hundreds of thousands of images of your everyday activities collected through a lifelogging device and you wanted to search for meaningful events in this collection? This is a novel area of research within the information retrieval domain~\cite{gurrin2014lifelogging}. A range of consumer products allow anyone to capture their daily activities and it is becoming increasingly popular~\cite{gurrin2014lifelogging}\cite{van2014future}\cite{askoxylakis2011log}. These devices are typically sold under the umbrella term of "lifelogging cameras". Like precursors blogging and vlogging, lifelogging is the next step in this series of life event recording devices. Lifelogging not only benefits consumers who wish to document their lives, but also has applications for security and policing services where there is a need to search the body cameras that are already being worn.

Preliminary research has provided an insight into how difficult searching lifelogging images is~\cite{scells2016qut}. The text-based search of images is most commonly referred to as TBIR (text-based image retrieval). This is what this research focuses on. The alternative, CBIR (content-based image retrieval) does away with textual features and relies on the visual features of an image; for example colour, shape or texture. In a study by Hartvedt~\cite{hartvedt2010using}, it was found that the benefit of TBIR over CIBR is that is supports image retrieval based on high-level textual semantic concepts. This claim is supported by Shao et.al~\cite{medical2004shao} who note that TBIR exploits semantic information in the text associated with images. CBIR is not applicable because images are searched with textual queries only and CBIR is not suited to this. TBIR is the predominant approach for image retrieval and most commercial search engines rely on this scheme~\cite{escalante2007towards}. TBIR intuitively relies heavily on quality annotations; if annotations have grammatical and spelling errors or do not correctly describe the content of the image (no semantic context) then the retrieval effectiveness will not be as good. The requirements of a `good' textual document in, for example, web search, also apply to annotations of images.

Currently, no machine learning approaches can consistently and reliably generate concepts (or annotations) from images. There are multiple methods of annotating images automatically, however this research is primarily focused on investigating the performance of manual annotations. The most appropriate type of annotation for text based image retrieval of lifelog images is discovered by investigating a number of annotation methodologies. These annotations are made available to the wider research community to further the research into the field of lifelog search. Supplementary data such as location, time, activity and personal health (heart rate, blood pressure, etc.) are important in pinpointing moments in time, however this research is focused on what types of annotations work best for textual search. Without meaningful annotations the supplementary information becomes useless when performing typical text-based image retrieval.

Collecting annotations is important, but another important challenge is to determine which annotation methodology is the best for text-based information retrieval. Evaluation of images generally relies on gold standard annotations, none of which exist for lifelog images. The problem becomes clear - the standard ways of evaluation are not applicable to lifelog annotations unless a reference exists. A framework involving TREC-style\footnote{\url{http://trec.nist.gov/}} runs is used for evaluating arbitrary types of annotations associated with lifelog images. The pipeline for evaluating lifelog images and their annotations and the data produced is an important and meaningful contribution to the research community. 

Due to the sheer volume of annotations required to collect in such a short time frame (six months) and with less than five annotators, automatic image captioning is also investigated as part of this research. An image captioning system will be trained on the manually collected annotations in order to annotate the entire collection of lifelog images. The NTCIR-12 lifelog semantic access task (LSAT) was a pilot task which required teams to retrieve and rank lifelog images using textual queries. This research task made a collection of images available for research purposes. The results from this research are compared with the results from the NTCIR-12 LSAT to determine if the methods of annotating improve over the performance of previous attempts. Runs are produced in the same format as the NTCIR-12 LSAT and evaluated under the same settings.

\section{Aims and Objectives}

Four annotation methodologies have been chosen as a basis for this research. Each methodology will involve the collection of annotations through some interface and evaluation. The aim of the project is to determine, through TREC-style evaluation, the best methodology for annotating lifelog images. The four methodologies under investigation are: 

\begin{enumerate}
    \item Tags --- Sets of keywords that describe objects and semantics of an image. Tags are chosen from a user-defined, non persistent vocabulary. These are similar to what one would expect to see on Flickr and other similar sites.
    \item Textual Descriptions --- Descriptive long-form annotations that contain semantic meaning. These are similar to the contents of documents such as web pages or news papers.
    \item Relevance Assessments --- Images are scored on a 0-10 scale by how relevant the image relates to a given concept. These are more purposely obtained, similar to editorial efforts by commercial search engine companies.
    \item Reverse Queries --- Queries are formulated for a given search result listing. Annotators are asked to provide a query that they think would result in the image being returned by a typical search engine. These are more akin to what one would expect to see in query logs.
\end{enumerate}

In order to collect annotations, the design and development of interfaces are required. This involves a website that facilitates the collection of these annotations. Once a suitable number of annotations have been collected, each methodology is evaluated. This involves a technique, whereby the evaluation of annotations is embedded within a search task (similar to evaluation in topic modelling). Each annotation methodology is evaluated in the same way --- the system does not change, only the type of annotation. This methodology can be applied or built upon for future work in the area of searching lifelog images.

This research also aims to produce two meaningful contributions to the lifelog research community. Recent results from the NTCIR-12 conference indicate lifelog search engines which incorporate image processing in some way perform much better than simply using annotations alone~\cite{safadilig2016ligmrim}. A new higher quality collection of annotations for lifelog images as well as training data for image classification systems is expected to boost the performance for everyone researching lifelog search engines. Secondly, anyone wanting to produce more annotations for a collection of lifelog images may wish to know how effective their annotation methodology is and how well it performs with respect to others.

Finally, automatic image captioning is investigated. Here, a state-of-the-art image captioning pipeline is generates captions for images using the manually collected annotations as training data. The end result is a fully annotated collection of lifelog images. Previous attempts at using textual descriptions as annotations~\cite{scells2016qut} resulted in poor, but optimistic results. The automatic approach to annotating all lifelog images in the collection distributed by NTCIR is compared to the previous attempt.

\section{Research Gaps}

The literature reviewed has revealed two gaps in the research. Annotation of lifelog images and the evaluation of them is not a standard process. Through preliminary research, a number of techniques have been investigated which are applicable to annotating images. A number of textual summarisation evaluation methods have also been discovered, however none of these are capable of evaluation without a ground truth or gold standard. Through researching the following two questions, it is hoped that these gaps in the research can be closed. According to an examination of evaluation in information retrieval systems~\cite[p. 24]{sanderson2010test}, there has been very little work done to evaluate how good a test collection is. Furthermore, finding all relevant documents to build topics is still the accepted approach for creating a test collection~\cite{cooper1973selecting}.

\section{Research Questions}

Three research questions have been identified by looking at previous research efforts in the literature. 

\textbf{Research Question 1:} How can annotations for a collection of lifelog images be evaluated?

It is important that the annotations of images are accurate and of a high quality. Low quality annotations will lead to poor evaluation results. Evaluation will be performed without using a gold standard annotation, since this does not exist. In this way, this research will use the ad-hoc TREC-style approach used in the NTCIR-12 lifelog semantic access task. This evaluation methodology is also referred to as extrinsic evaluation, whereby annotations are evaluated by determining the performance of a larger system as a whole with respect to each annotation methodology.

\textbf{Research Question 2:} What are the possible ways lifelog images can be annotated and what is their retrieval effectiveness?

There are many state of the art solutions for automatically summarising the contents of images~\cite{karpathy2015deep}\cite{jia2014caffe}\cite{pan2004gcap}, however this project involves measuring the effectiveness of manual annotations. This is due to the fact that there does not exist any training data specifically for lifelog images. This makes it challenging to train a model that can identify the contents of a lifelog image. It is thought that by manually annotating images, a well formed collection of annotated lifelog images can be produced, in addition to data which machine learning and computer vision algorithms can exploit.

\textbf{Research Question 3:} RQ3: How do annotations generated by current state-of-the-art automatic image captioning techniques compare to the effectiveness of manual annotations?

\todo{todo}