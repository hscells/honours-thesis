\chapter{Future Work}

\todo{this section is just notes for now, I need to expand and gather my ideas, and also write it into paragraphs.}

There has been some work on generating images based on textual descriptions, but it looks limited to small collections of images, and does not seem to be able to produce large resolution images. In the future, search may be able to be performed by image description->image generation->image matching \todo{this https://github.com/paarthneekhara/text-to-image and the paper should be cited here}
 
Current state of the art automatic image captioning algorithms~\cite{karpathy2015deep} allow many images to be textually annotated with some level of accuracy. 
 
The collection of images used is still relatively small. Although diverse in moments and objects, there are still less than 100k images. A large-scale test collection in the magnitude of hundreds of thousands of images to a million or more images could offer better results. \todo{I have read some discussions on line about this but I am yet to find any papers/formal research on this, as well as the exact number of images}. Results will only get more accurate because there is a more diverse range of events happening in the collection.

There are four annotation types under investigation. Each one is very different to the last. There have been no studies into the most effective methodology or set of methodologies for annotating lifelog images. It would be nice to investigate more than four, however due to time constraints this is inconceivable.