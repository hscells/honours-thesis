\documentclass[12pt,a4paper]{book}
\usepackage[titletoc,toc,title]{appendix}
\usepackage{color}
\usepackage{colortbl}
\usepackage{graphicx}
\usepackage{algpseudocode}
\usepackage{graphicx}
% \usepackage{url}
\usepackage{verbatim}
%\usepackage{caption}
%\usepackage{subcaption}
%\usepackage{textcomp}
%\usepackage{fancyvrb}
\usepackage{graphicx}
\usepackage{amssymb}
\usepackage{epstopdf}
\usepackage{hyperref}
\usepackage{float}
\usepackage{adjustbox}
\usepackage{subfig}
\usepackage{placeins}
\usepackage{multirow}
%\usepackage[square]{natbib}
\usepackage[nottoc]{tocbibind}
\usepackage[pass]{geometry}

\definecolor{Gray}{gray}{0.8}
\definecolor{LightRed}{rgb}{0.99,0.9,0.9}

\newcommand\todo[1]{\textcolor{red}{#1}}

\begin{document}
\title{\small Thesis\\\huge Investigating Methods Of Annotating Lifelogs For Use In Search}

\author{Harry Scells\\harrisen.scells@connect.qut.edu.au\\\\\small Supervisor - Guido Zuccon\\\small g.zuccon@qut.edu.au\\}
\maketitle

\chapter*{Abstract}
The notion of `quantified self' in recent years is allowing people to more easily capture data about themselves. Lifelogging is an umbrella term which encompasses multiple personalised data gathering forms which primarily involve wearing a device that continuously records images of every day events via one minute snapshots. These data capturing processes create vast amounts of personalised data about the user, however there is a lack of effective ways to search this data that can accurately retrieve moments of significance that a user may be interested in. The primary form of lifelogging is through the capture of images from wearable cameras. While image search has been the subject of extended research in the past and great advances have been made in both text-based image retrieval and content-based image retrieval only limited work has considered searching lifelog image data, and the solutions currently available have not demonstrated effective performance. In this thesis four annotation methodologies are investigated to discover their performance in a text-based image retrieval system. These methodologies include textual descriptions of the images, tags, weighted tags, and queries. Annotations are investigated to understand which methodology is the best to enable effective text-based image retrieval on lifelog data and then automatic image captioning is investigated to determine if textual annotations can automatically be derived. Finally, each methodology is compared in terms of the cost to manually and automatically collect annotations.

The results of this research indicate that under ideal conditions, annotations which represent a query are the most effective in a retrieval task. This is compounded by the fact that on average they are the most cost effective annotation to collect of the four under investigation. The query annotation represents the text a user may enter into a typical search engine to retrieve an image they are looking for.

\chapter*{Acknowledgements}
Thank you to my friends and family for the massive amount of support and encouragement over the past year -- I could not have done this without it!

Most importantly I would like to thank my amazing supervisor Guido for the incredible opportunities and support he has provided me with -- and for motivating me to pursue honours in the first place. It has been a challenging but amazing year of learning, and I am very grateful.


\tableofcontents

\chapter{Abstract}

\chapter{Introduction}

What if you had hundreds of thousands of images collected passively throughout the day and you wanted to search for events in this collection? This is a brand new, novel area of research within the domain of information retrieval~\cite{gurrin2014lifelogging}. A range of consumer products allow anyone to capture their daily activities and it is becoming increasingly popular~\cite{gurrin2014lifelogging}\cite{van2014future}\cite{askoxylakis2011log}. The devices are typically sold under the umbrella term of "lifelogging cameras". Like blogging and vlogging before it, lifelogging is the next step in this series of life event recording devices. Not only is this an area of research for consumer products, but also has applications for security and policing services where there is a need to search the cameras which are already being worn by members of these fields.

Preliminary research has provided an insight into how difficult searching lifelogging images is. Typical search methodologies are not applicable and, until recently, there have been no freely available collections. Currently, there are no machine learning approaches which can consistently and reliably discover concepts and objects from images. Since the best option is still to annotate images manually using humans, this is what this research aims to assist with. By investigating a number of annotation methodologies and discovering which one most well suits lifelog images, we hope to provide an answer for the best way of annotating a collection of lifelog images. It is hoped that these annotations can be contributed to the wider research community to further the research into the field of lifelog search. Supplementary data such as location, time, activity and personal health (heart rate, blood pressure, etc.) are important in pinpointing moments in time, however this research will focus on what types of annotations work best for textual search. Without meaningful annotations the supplementary information becomes useless when performing typical text-based search.

This text-based search of images is most commonly referred to as TBIR (text-based image retrieval). The alternative, CBIR (content-based image retrieval) does away with textual features and relies on the visual features of an image; for example colour, shape or texture. The benefit of TBIR over CIBR is that is supports image retrieval based on high-level semantic concepts~\cite{hartvedt2010using}. While it is believed by~\cite{hartvedt2010using} that neither TBIR nor CBIR alone are optimally suited to support context focused image retrieval, the different annotation methodologies aims to prove or disprove this belief.

Collecting annotations is important, but determining which methodology is the best is a larger challenge. Typical textual evaluation strategies rely on a gold standard annotation. The problem becomes clear - these standard ways of evaluation are not applicable to lifelog annotations. There are currently no ways of evaluating arbitrary types of annotations for lifelog image. Designing a system for evaluating lifelog images and their annotations, and not only the types of annotations being researched by this project, but any type of annotation, will also be an important and meaningful contribution to the research community. 

Due to the sheer number of annotations needed to collect in such a short amount of time, automatic image captioning will also be investigated as part of this research. An image captioning system will be trained on the manually collected annotations in order to annotate the entire NTCIR-12 collection of lifelog images. We can then compare the results from this research with the results from the NTCIR-12 lifelog semantic access task and see if the methods of annotating improve the performance of previous attempts.

\section{Aims and Objectives}

Four annotation methodologies have been chosen as a basis for this research. Each methodology will involve the collection of annotations through some interface and evaluation. The aim of the project is to determine, through evaluation, the best methodology for annotating lifelog images. The four methodologies under investigation are: 

\begin{enumerate}
    \item Tags --- Sets of keywords that describe objects and semantics of an image. Tags are chosen from a user-defined, non persistent vocabulary.
    \item Textual Descriptions --- Descriptive long-form annotations that contain semantic meaning.
    \item Relevance Assessments --- Images are scored on a 0-10 scale by how relevant the image relates to a given concept.
    \item Reverse Queries --- Queries are formulated for a given search result listing. Annotators are asked to provide a query that they think would result in the image being returned by a typical search engine.
\end{enumerate}

In order to collect annotations, an interface will need to be designed and developed. This will involve a website that will facilitate the collection of these annotations. Once a suitable number of annotations have been collected, each methodology will be evaluated. This will involve a novel technique, whereby the evaluation of annotations is embedded within a search task (similar to topic modelling). The evaluation will be generic, in that \textit{any} type of annotation can be submitted for evaluation. It is hoped that the methodology can be applied or built upon for future work in the area of searching lifelog images.

This research also aims to produce two meaningful contributions to the lifelog research community. Recent results from the NTCIR-12 conference indicate lifelog search engines which incorporate image processing in some way perform much better than simply using annotations alone. A new collection of \textit{well} annotated images as well as training data for image classification systems would hopefully boost the performance for everyone researching lifelog search engines. Secondly, anyone hoping to produce more annotations for a collection of lifelog images may wish to know how effective their annotation methodology is and how well it performs with respect to others. An easy to use tool for evaluation of lifelog image annotations is a necessity for this new area of research.

Finally, automatic image captioning will be investigated using the manual annotations that will be collected. Here, a state-of-the-art image captioning tool will be used to annotate images using the annotations that have already been collected. The end result will be a fully annotated collection of lifelog images that can be searched and distributed also. Previous attempts at using textual descriptions as annotations~\cite{scells2016qut} gave poor, but hopeful results. It is hoped that by simply adding more annotations, the performance of previous search engines can be improved.

\section{Research Gaps}

The literature reviewed has revealed two gaps in the research. Annotation of lifelog images and the evaluation of them is not a standard process. Through preliminary research, a number of techniques have been investigated which are applicable to annotating images. A number of textual summarisation evaluation methods have also been discovered, however none of these are capable of evaluation without a ground truth or gold standard. Through researching the following two questions, it is hoped that these gaps in the research can be closed. According to an examination of evaluation in information retrieval systems~\cite[p. 24]{sanderson2010test}, there has been very little work done to evaluate how good a test collection is. Furthermore, finding all relevant documents to build topics is still the accepted approach for creating a test collection~\cite{cooper1973selecting}.

\section{Research Questions}

Two research questions have been identified by looking at previous research efforts in the literature. 

\textbf{Research Question 1:} How can annotations for a collection of lifelog images be evaluated? It is important that the annotations of images are accurate and of a high quality. Low quality annotations will lead to bad evaluation results. Evaluation will be performed without using a gold standard annotation, since this does not exist. In this way, this research will investigate a novel way of evaluating annotations for lifelog images. The most promising way to do this is through extrinsic evaluation, whereby annotations are evaluated by determining the performance of a larger system as a whole with respect to each annotation methodology. In order to keep evaluation fair, a number of baseline retrieval models will be tested to weigh out the differences between them.

\textbf{Research Question 2:} What are the possible ways lifelog images can be annotated? There are many state of the art solutions for automatically summarising the contents of images, however this project will involve manual annotations. This is due to the fact that there does not exist any training data specifically for lifelog images. This makes it challenging to train a model that can identify what is in a lifelog image. It is thought that by manually annotating images, a well formed collection of annotated lifelog images can be produced, as well as data which machine learning and computer vision algorithms can exploit.

\chapter{Background}

This chapter presents a background on the current state of searching lifelog images. This is done through a review of the literature and findings from the NTCIR-12 lifelog semantic access task.

% ---------------------------------
%  BEGIN LIT REVIEW
% ---------------------------------
\section{Literature Review}

\subsection{Representing the Data}
One way of viewing lifelogging is the process of creating a surrogate memory for a person. Organising and presenting are the key challenges for lifelog search engines. Gurrin et al.~\cite{gurrin2014lifelogging} proposes that it is possible to segment the raw, unprocessed lifelog data into meaningful units, or events, which he defines as: "a temporally related sequence of lifelog data over a period of time with a defined beginning and end". In order to perform information retrieval, the events need to be annotated with meaningful semantics. Annotations can either be manually created by humans, or generated through machine learning algorithms. These annotations must also be evaluated for effectiveness, as poor annotations will lead to poor performance when performing retrieval on the images.

There are five aspects of human memory access, as proposed by Gurrin et al.~\cite{gurrin2014lifelogging}:
\begin{enumerate}
    \item Recollecting: Concerned with re-living and accessing past experiences of episodic memories.
    \item Reminiscing: A form of recollecting, concerned with reliving past experiences for emotional or sentimental reasons.
    \item Retrieving: A more specific form of recollecting  in which specific information needs are to be retrieved such as an address, a document, a location, or any atomic piece of information.
    \item Reflecting: A form of quantified-self analysis, performed in order to discover knowledge and insights that may not be immediately obvious.
    \item Remembering: Concerned with prospective memory more than episodic memory. A form of planning for future activities or to act as a reminder or prompt for tasks that a person would like to do.
\end{enumerate}
Gurrin et al.~\cite{gurrin2014lifelogging} argues that an information retrieval system targeted at lifelogging should focus on the Five R's as information needs for the user.

In the context of searching images on the web,  L. Vuurpij, et al.~\cite{vuurpij2002vind} states that image retrieval systems are restricted to the domain they cover and require a lot of domain knowledge in order to fulfil the information needs of a user. Furthermore, L. Vuurpij, et al.~\cite{vuurpij2002vind} notes that there has been a shift from computer vision and pattern recognition to psychology and cognitive science in the domain of image retrieval, where models like the Five R's~\cite{gurrin2014lifelogging}, are becoming more prevalent. 

\subsection{Annotating Lifelog Images}
The current state of the art models for image retrieval use tag-based or textual description annotations \cite{ali2010semantically}. This is typically due to the fact that retrieval models can use this text as a bag of words, or use the text to attribute some form of semantic meaning.

\subsection{Evaluating Annotations}
Without high quality annotations, semantic search would not work, since semantic search exploits the meaning and context of a sentence rather than the keywords in it \cite{ali2010semantically}. The semantic and contextual data associated with images is important for an effective retrieval model that uses textual features rather than pixel data in images. This textual data can either be generated using machine learning algorithms, as demonstrated by A. Karpathy et al.~\cite{karpathy2015deep} and  I. Sutskeve et al.~\cite{sutskever2011generating}, or generated manually by humans. The machine learning algorithms do however start from test data, typically of a specific domain or a range of domains, which they learn from. It is important to note that this can have undesirable consequences when trying to apply a model which has been trained on one domain to one which it has no knowledge of. In both situations, it is essential that the annotations themselves are evaluated such that they describe the image with enough detail and are convincing to humans, since queries will be formulated by humans. Three widely used models for this exist: BLEU\footnote{Bilingual Evaluation Understudy} \cite{papineni2002bleu} which is precision based, ROUGE\footnote{Recall-Oriented Understudy for Gisting Evaluation} \cite{lin2004rouge} which is recall based, and METEOR\footnote{Metric for Evaluation of Translation with Explicit ORdering} \cite{elliott2013image}, used for judging the overall quality of annotations.

All of the metrics above were initially proposed with respect to the evaluation of automatic summarisation and natural language processing. Furthermore, they all use a reference annotation in order to score annotations. ROUGE compares the number of overlapping n-grams, word sequences, and word pairs of annotations with ideal annotations created by humans \cite{lin2004rouge}. BLEU counts the maximum number of times a word appears in any reference annotation, followed by "clipping" the total count of each candidate word by its maximum reference count, adding these "clips" up, and dividing by the total unclipped number of candidate words in the annotation \cite{papineni2002bleu}. The notion of "clipping" in BLEU is a variation of precision whereby words are only accepted for the maximum number of times they appear the reference text, for instance if a word appears in an annotation five times but is in the reference annotation twice, the "clipped" value would be 2/5. Finally, METEOR generally operates by unigram matching (bag of words) between a reference annotation, typically created by a machine, and a human produced annotation. Both METEOR and ROUGE take multiple approaches to comparing annotations, for reference, ROUGE:
\begin{enumerate}
    \item ROUGE-N N-gram Co-occurence Statistics
    \item ROUGE-L Longest Common Subsequence
    \item ROUGE-W Weighted Longest Common Subsequence
    \item ROUGE-S Skip-Bigram Co-Occurence Statistics
    \item ROUGE-SU Skip-Bigram Co-Occurence Statistics with Unigram\\ Counting Unit
\end{enumerate}
The unigrams matched in METEOR can be based on surface forms, stemmed forms, and meanings, with the option to be extended \cite{elliott2013image}.

R. Vedantam et al.~\cite{vedantam2015cider} argue through the results of their experimentation, however, that there exists a more effective model for evaluation which is rooted in human consensus. Their method, CIDEr (Concensus-based Image Description Evaluation) is a model which outperforms all other models of evaluating descriptive annotations of images. CIDEr performs so well due to high correlation with human judgement and consensus. According to R. Vedantam et al.~\cite{vedantam2015cider}, the CIDEr metric inherently captures sentence similarity, the notions of grammatically, salience, importance (precision), and accuracy (recall). CIDEr appears to improve upon the other three models and takes into account the weaknesses the other models may have, however it still relies on reference annotations.

The evaluation methods above rely on the availability of a ground truth or reference annotation that can be used to compare with the automatically generated annotation. This approach, however, is ill-suited to generating annotations for lifelog images as it is unclear what these annotations should ``look like'', because it is unknown what makes an annotation of a lifelog image ``good''. A better suited alternative for this problem is to embed the evaluation of different annotation methods within a task and thus evaluate the  methods with respect to the effectiveness the different methods induce on the task. Specifically, in this research project, the aim is to embed the evaluation of lifelog annotation within a search task. Thus the effectiveness of a system would be evaluated with respect to the search task. None of the system properties would vary apart from the method that is used to annotate images.This evaluation methodology is akin to, for example, previous work that has examined the effectiveness and quality of different topic modelling techniques and semantic models via evaluating the effect they have on search engine result effectiveness~\cite{wei2006lda,zuccon2015integrating,karimzadehgan2010estimation,yi2009comparative}.

\subsection{Types of Annotations}
As suggested by R. Yan et al.~\cite{yan2008learning}, the most common image annotation approaches can be categorised into two types. The first is \textit{tagging}, where annotators choose a set of keywords from a vocabulary for each image. The second most common approach is described as \textit{browsing}, where a group of images are judged against the relevance of a predefined keyword. There are, however, less commonly used annotation approaches, for example, \textit{descriptive natural language annotations} which are generated in a model by A. Karpathy et al.~\cite{karpathy2015deep}. This model outperforms the previous work done in this area of research for both image retrieval and image annotation on the Flickr8K, Flickr300K and MSCOCO. B. Hu et al.~\cite{hu2003ontology} clarifies why high quality textual descriptions generally perform better than systems that employ keyword or tag based annotation models, in that these models suffer from several limitations:
\begin{enumerate}
    \item A keyword in a document does not necessarily mean that the document is relevant
    \item A relevant document may not contain the explicit word
    \item Synonyms of the query keywords lower the recall rate (ratio of retrieved images which are relevant to the total number of relevant images, see Appendix A for details)
    \item Homonyms of the query keywords lower the precision rate  (ratio of relevant images that are successfully retrieved to the total number of relevant and irrelevant images retrieved) see Appendix A for details)
    \item Semantic relations such as hyponymy, meronymy, antonymy are not exploited
\end{enumerate}

Recent work in the consumer health search domain by Zuccon et al.~\cite{quteprints82599} and Stanton et al.~\cite{stanton2014circumlocution} focused on generating queries from images. The aim of their research was to understand how the general public would search for information if they had a medical condition as that in the image presented to them. This new methodology used by these previous works could be adapted to the context of gathering annotations for lifelogging, thus leading to an \textit{annotation by querying} method. This method would consist of showing annotators an image from a lifelogger and ask them to provide the queries they would issue to a (standard) search engine to attempt to retrieve the image itself.

\subsection{Searching for Lifelog Images}
Lifelog information retrieval systems typically have very poor performance due to there not being any formal models made specifically for the field, as reported by Gurrin et al.~\cite{gurrin2014lifelogging}. Until very recently, there have been no large, distributable test collections such as the TREC collection for text \cite{gurrin2014lifelogging}. The NTCIR collection is a set of tagged lifelog images which have been collected by researchers who wore a lifelogging camera for a short amount of time \cite{gurrin2016ntcir}. The tags were automatically generated by using a pre-trained image tagging algorithm.

While there is limited applied methodology to retrieval models in lifelogging, there has been much discussion about what the models should try and solve. Both H.  W.  Bristow et al.~\cite{bristow2004defining} and A. R. Doherty et al.~\cite{doherty2010automatically} corroborate that detecting and interpreting the implicit semantics and context of lifelogging data from heterogeneous sources would be advantageous in explaining the Who?, What?, Where? and When? questions which occur in every day events. It was also noted that these questions are common among image searchers and that they are not capable of being answered by normal indexing like that in traditional search engines \cite{ali2010semantically}.

While there has been some research into tagging and annotating images, there has not been as much work in developing a model for searching these images within the context of lifelogging \cite{gurrin2014lifelogging}. Typical image search engines for web pages treat the surrounding text, captions, alternate text and HTML titles \cite{frankel1996webseer} as a bag of words for retrieval. The success of these search engines rely on a sufficient amount of surrounding text, something which is not provided by current automatic image annotation models for lifelogging. The longer and more detailed the text is within the context of the image, the better the performance of the search engine. This is perhaps why other research has involved novel search techniques \cite{vuurpij2002vind}, since the current models for generating captions of images are not yet detailed or accurate enough for current textual information retrieval models to work.

Generally, image based retrieval methods can be classified into two categories: text-based image retrieval (TBIR) and content-based image retrieval (CBIR). A CBIR system utilises image features such as grid colour movements, edge direction histogram, Gabor textual features, and Local binary pattern histograms. as described in work by Wu et al.\cite{wu2009distance}. These features (colour, texture, shape, SIFT keypoints) become a query to the search engine which match visually similar images. CBIR systems, although extensively studied for over a decade, are still limited in comparison to TBIR systems. Zue et al.\cite{zhu2010image} provide three points for why this is:
\begin{enumerate}
    \item The semantic gap that exists between low-level visual features and high-level semantic concepts
    \item The low efficiency due to high dimensionality of feature vectors
    \item The query form is unnatural for image searching (appropriate example images may be absent)
\end{enumerate}
The efficiency of TBIR can be explained when one considers that it can be formulated as a document retrieval problem and can be implemented using the inverted index technique. The downside to TBIR is that is highly expensive: experimental evidence by Wu et al.~\cite{wu2013tag} shows that the performance of TBIR is highly dependent on the availability and quality of manual annotations. If this process can be automated and images can be automatically captioned, it would solve a fundamental issue that exists with TBIR systems.

\subsection{Automatically Captioning Images}

There have been some recent advances in machine learning which combine convolutional neural networks and recurrent neural networks that enable images to be automatically captioned. In this thesis neuraltalk2~\cite{karpathy2015deep} is used for automatically captioning images. Neuraltalk2 works by  feeding the last hidden layer of the convolutional neural network (CNN) as input into the recurrent neural network (RNN). This recipe is followed by other current state of the art automatic image such as Jia et al.~\cite{jia2014caffe} (CAFFE\footnote{Convolutional Architecture for Fast Feature Embedding}), and Vinyals et al.~\cite{arXiv2016160906647V}.

These neural network approaches have reported to perform better than previous systems. Older systems attempt to manually select features and perform some clustering or probabilistic technique. Some examples of these previous works include a subspace clustering algorithm by Wang et al.~\cite{wang2004automatic} and a probabilistic approach that involves graphs by Pan et al.~\cite{pan2004gcap}. Wang et al. identifies the core problem with manually selecting features: that image data is highly dimensional, and many dimensions are irrelevant which confuse these older systems (i.e. hiding clusters in noisy data). This is why neural networks and deep learning excel at captioning --- they learn the features and how to optimally segment images.

Generating captions for images reduces the time cost of TBIR systems, although manually annotating or labelling a test collection for training data still takes time and is prone to human error. This process may be able to be alleviated by automatically generating images from text. Recent machine learning architectures like that of Reed et al.~\cite{reed2016generative}, while limited to specific domains, can produce images from textual descriptions. In time this could allow hybrid TBIR/CBIR systems that outperform the current state of the art image retrieval systems.

Attempts have been made at image retrieval that exploits both TBIR and CBIR methodologies. One such attempt by Escalante et al.~\cite{escalante2007towards} introduces two novel formulations: annotation based expansion (ABE) and late fusion of heterogeneous methods (LFHM). In ABE, segmented regions in images are labelled. An annotation is formed by expanding the labels into a textual representation. The annotation associated with an image is treated like a document and typical text-based retrieval is then performed. LFHM consists of manually building several retrieval models based on different information from the same collection of documents. Each retrieval model returns a list of relevant documents to a query and the output of each is combined to obtain a single list of ranked documents.

% ---------------------------------
%  END LIT REVIEW
% ---------------------------------

\section{Findings From NTCIR}
The NTCIR-12 lifelogging latent semantic access pilot task consisted of four participants contributing to the automatic retrieval component and one participant in the interactive retrieval component. The highest performing automatic team, LIG-MRIM, used computer vision to classify images and not rely on the visual concepts distributed with the task. The interactive team (LEMoRe) outperformed all other automatic teams, but this is expected as is generally the case with tasks that contain both automatic and interactive components. The other three automatic teams are VTIR, III\&CYUT and QUT (the preliminary work done for this research).

LIG-MRIM~\cite{safadilig2016ligmrim} used dynamic convolutional neural networks and a multi-class support vector machine (MSVM) in order to classify images. Visual indexing is composed of two parts: three deep convolutional neural network models (AlexNet, GoogleNet and Visual Geometry Grouping (VGG)) process each image. The output is normalised and has principal component analysis (PCA) performed on it. These outputs are then concatenated together. The same normalisation and PCA process is repeated and fed into the MSVM. The output of this is concatenated with the VGG data. The second part involves temporally naming times of the day in order to attempt to extract semanitic meaning from times of the day. While this team submitted runs that fit the definition of automatic for the task, queries were generated manually from the topics by an expert.

III\&CYUT~\cite{lin2016image} used a traditional textual based approach to lifelog retrieval. A skipgram word embedding obtained with word2vec\cite{mikolov2013word2vec} was computed for the visual concepts dis- tributed with the dataset. These embeddings were then used in an attempt to add more semantic meaning to images. Specifically, the embeddings were used within a document expansion process, resulting in a translation language model~\cite{zuccon2015integrating}. Query expansion was also used on every keyword.

The VTIR team~\cite{xia2016vtir} attempted to exploit location metadata associtaed with the images. To this end 3,000 random images were labelled against a rich semantic location ontology. More concepts were utilised by applying the WordNet database to find cognitive synonyms. Despite the additional annotations, this system failed to provide good retrieval effectiveness.

Finally, the LEMoRe~\cite{de40lemore} team, which used an interactive approach, combined existing technologies and methodologies in order to develop a search engine. Colour correlogram, edge histogram, joint composite descriptor and pyramid histogram of oriented graphics are used by the image retrieval system as features to retrieve images. Both a novice and an expert used the system to produce runs.

\begin{figure}[h]
    \centering
    \includegraphics[width=0.95\textwidth]{graphs/ntcir-pr-curve}
    \caption{Precision-recall curves for the four LSAT teams}
    \label{fig:ntcir-results}
\end{figure}

The results from this pilot task offer a promising glimpse into the future of searching lifelog images. Figure \ref{fig:ntcir-results} presents the best run from each of the teams that participated in NTCIR-12 LSAT. LIG-MRIM shows that automatic methods for annotating lifelog images can result in decent text-based image retrieval effectiveness. The teams that do not perform well (including the QUT team) offer insight into areas of research to avoid. 

The difference in retrieval effectiveness between LIG-MRIM and the other three teams is highly likely due to the annotations of the images. The task provides teams with automatically generated annotations for the images distributed with the task. These annotations are generated using a previously state-of-the-art captioning framework, CAFFE~\cite{jia2014caffe}. The problem lies within the fact that a CAFFE model is trained on a data set that does not align with the lifelog images. Three of the teams use these annotations in their systems; however LIG-MRIM generate their own annotations. This indicates that no matter how well tuned and suited a text-based image retrieval system is to lifelog images, poor textual representations for images can significantly impact the retrieval effectiveness.
\chapter{Methods}

\section{Sampling Images}

The number of images in the NTCIR-12 Lifelog data set far exceeds the possible number of images that can be annotated manually. The collection is very large, containing around 90,000 images. It is certainly unfeasible to annotate every image four times (for each annotation methodology), so sampling images to reduce the number of images to annotate is necessary. There are two methodologies employed to sample images: The first is based off previous work~\cite{scells2016qut} which identifies a way to cluster lifelog images using image histograms for features and aligning the images temporally to determine cluster segmentation points. After clustering images, a this sampling technique involves selecting one image at random from each cluster. The cosine similarity measure is used to determine if an image should be added to an existing cluster. The threshold of the cosine value between two images is set to 0.86\footnote{This value was empirically found to provide a range of sufficiently different clusters, each containing similar images}. Clusters are then combined based on visual similarity using the aforementioned image histograms and a representative image from each of these clusters is chosen at random. This process results in about 16,000 images chosen for annotation. The second sampling method entails processing the qrels to extract only the relevant images. A maximum of 30 images are selected from each topic which results in just over 1,000 images required for annotation (since some topics have less than 30 relevant images, some even having less than 10).
% Given more time, it would be worthwhile investigating other image similarity measures (such as those used by the LEMoRe team at NTCIR-12~\cite{de40lemore}) to possibly produce a more uniformly distributed sample set.

There is some overlap between the images chosen from the clustering process and the images that are known to be relevant, however it reduces the number of images to annotated from 90,000 down to around 16,000. In reality all of the images which are known to be relevant are annotated and around 1,000 from the clustered set of images are annotated. Annotating both relevant and non-relevant images ensures that retrieval is working correctly; both relevant and non-relevant images should be retrieved but the images annotated with relevant annotations should rank higher.

These sampled images are uploaded into a database to be annotated. Annotations are then collected through specially designed interfaces.
% Both sampling techniques are required 
% For future work, a combination of clustering and matching images to relevant topics might be a better option, as there will most likely be many irrelevant annotations. There may also be no annotations which are relevant to a topic as well, since the sampling is somewhat random. This makes the process less than idea, but will serve as a proof of concept for further research.

\section{Collecting Annotations}

Once the images have been sampled and uploaded into a database, they are ready to be annotated. Four web application interfaces are used for the collection of annotations. Each interface provides a means for an annotator to assign the respective annotation type to an image from a list of unannotated images. The architecture of these interfaces consists of:
\begin{enumerate}
    \item A database to store the annotations and the sampled images. The database also stores information about each user performing the annotations, who annotated which image and how long it takes to do an annotation. This ensures there is a record of who annotated each image, and allows for an analysis to be performed at a later stage.
    \item A web server and that handles the `business logic'. This web server exposes some password protected RESTful services that applications can hook into.
    \item A website which consumes the API provided by the web server and handles `view logic'. This is the layer that annotators will interact with directly. Each interface will be one of these views.
\end{enumerate}

Each interface is designed with great care to ensure the collection of annotations is as painless as possible. It is important that the time between annotating one image and starting to annotate the next is as small as possible. If it takes a minute to annotate one image, it will take an hour to annotate only 60 images.

It is for this reason that an automatic captioning system is used to caption the rest of the images. For each methodology, annotations will be generated using a state-of-the-art machine learning architecture. This ensures every single image is annotated, so an annotation has been collected for each image, for each annotation methodology.

\subsection{Annotation Methodologies}

Four annotation methodologies are selected for investigation: \textbf{textual}, \textbf{tags}, \textbf{relevance assessment} and \textbf{reverse query}. While the annotation types are wildly different to each other, they are all  collected in a very similar way. An expert annotator is shown a randomly chosen image from the sampled set and is asked to provide an annotation (or in the case of relevance assessment, multiple annotations) for the image. The interfaces used for collecting annotations are pictured and described as follows:

\textbf{Textual}

\includegraphics[width=0.95\textwidth]{images/text-interface}

Here, annotations are collected using free form text through a text box on the page. Textual annotations should contain semantic information about an image, and should describe the image with as much detail as possible. These annotations are very similar to a textual document in a typical web search engine, which is why they are selected as one of the methodologies to investigate.

\newpage
\textbf{Tags}

\includegraphics[width=0.95\textwidth]{images/tag-interface}

Tags are collected through a specifically designed interface. These vocabulary of tags is created from previously added tags, meaning the list of tags available is arbitrary and can be expanded. 

\newpage
\textbf{Reverse Query}

\includegraphics[width=0.95\textwidth]{images/query-interface}

In this interface, user queries are collected by presenting an image taken from the lifelog camera and asking the annotator to provide a query with what they expect to be returned by a typical search engine. This is a relatively new and novel way to annotating \textit{any} type of document or image~\cite{quteprints82599}. 

\newpage
\textbf{Relevance Assessment}

\includegraphics[width=0.95\textwidth]{images/rel-ass-interface}

Relevance assessment involves presenting an annotator with an image, and asking them to judge how relevant a concept is to the image. Assessors are asked to choose concepts from a list to assess, and from those chosen concepts are asked to assess the relevance of that concept to the target image. Concepts are ranked on a scale of zero to ten, where zero is not relevant at all and ten is highly relevant.

These annotations will be collected last, since the list of concepts is formed after analysing the terms. Concepts are chosen by creating a list of terms from the existing textual, tag and query annotations, which are then filtered down to the terms that occur in the NTCIR-12 Lifelog topic titles and descriptions. Each term is then scored using IDF\footnote{Inverse Document Frequency} and then ranked using an algorithm similar to discounted cumulative gain~\cite{jarvelin2002cumulated}; in which higher scoring terms are selected less frequently, and more emphasis is placed on selecting lower scoring terms. 

\newpage
\textbf{Automatic Image Annotation}

Finally, an attempt to automatically caption images for each annotation methodology is done by exploiting a recent state-of-the-art machine learning image captioning approach~\cite{karpathy2015deep} which has been open sourced\footnote{\url{https://github.com/karpathy/neuraltalk2}}. The architecture of the system consists of the final hidden layer of a convolutional neural network (CNN) which learns features of image regions being fed into a recurrent neural network (RNN) that generates textual descriptions.

A model is trained using 70,000 iterations using the $adam$ optimiser with $\alpha$ set to 0.8 and $\beta$ set to 0.999. The learning rate of the language model is set to 0.0004. More iterations are performed afterwards to fine tune the deep learning architecture, but generally insignificant changes if any are seen in the values of the optimisation function.

\section{Evaluating Annotations}

Annotations are evaluated with an ad-hoc, TREC style methodology. The topics and qrels from the NTCIR-12 Lifelog semantic access pilot task are used to perform evaluation. In total eight runs are produced: four consisting of the manually annotated annotations and another four containing the combination of the manually and automatically annotated annotations. This is to find the best annotation methodology - but also to see if automatic annotations can increase the retrieval effectiveness.

Document rankings are generated by submitting queries to ElasticSearch using porter stemmer and a default English stoplist. Queries are formulated by using the title field from the NTCIR-12 Lifelog topics. The stemming and stopping are also applied to the queries. ElasticSearch is also set to only retrieve a maximum of 1,000 images. The runs produced by this system are evaluated using \verb|trec_eval| and the NTCIR-12 Lifelog qrels.

The document rankings and evaluation is performed through a custom-built framework. A Java RESTful application wraps Elasticsearch. This is so a query issued to the Java application can return a properly formatted TREC run, and so many queries can be issued with one request. 

\chapter{Results}\label{chapter:results}

The findings of the research is presented in this chapter in the form of the annotation statistics and the retrieval effectiveness of the annotations. The annotation statistics cover the time taken to annotate using each annotation methodology, and details about the number of annotations collected. The retrieval effectiveness section provides a breakdown of the performance of both the manually and automatically collected annotations.

\section{Annotation Statistics}

In total, five annotators managed to annotate a total of 10,982 images across all annotation methodologies. These annotators consist of students and staff from an information retrieval group at QUT. Figure \ref{fig:annotator-breakdown} illustrates the number of annotations completed by each annotator. Two annotators account for the majority of the annotations, while three others provide an additional 904. The exact number of each annotation type, the total time it took to annotate each annotation type and the average time it took to annotate is displayed in Table \ref{table:annotation-stats}.

\begin{table}[b]
    \centering
    \begin{tabular}{ | l | l | l | l | p{5cm} |}
    \hline
    Name & Count & Average Time & Total Time \\ \hline
    Text & 3172 & 1 minute & 2 days, 23 hours \\ \hline
    Tag & 2897 & 31 seconds & 23 hours, 40 minutes \\ \hline
    Query & 3616 & 16 seconds & 15 hours, 10 minutes \\ \hline
    Assessment & 1327 & 58 seconds & 21 hours, 36 minutes \\ \hline
    \end{tabular}
    \caption{Annotation statistics obtained by taking the average across all five annotators}
    \label{table:annotation-stats}
\end{table}

\begin{figure}[t]
    \centering
    \includegraphics[width=0.8\textwidth]{graphs/annotator-breakdown}
    \caption{Total number of annotations by annotator}
    \label{fig:annotator-breakdown}
\end{figure}

A statistical analysis of the collected annotations reveals that they are appropriate for a typical textual corpus. Figures \ref{fig:idf-scores} and \ref{fig:tf-scores} are what one would expect to see in a Zipfian distribution~\cite{tullo2003modelling}; that is the frequency of each concept is inversely proportional to it's rank in the frequency distribution (the most commonly used concept appears twice as often as the second most frequent concept and three times as often as the third most frequent concept). The notion of concepts are different to terms since a concept can contain more than one word (this is possible through tags, where a tag can contain multiple words such as `shopping mall' and `street sign').

\begin{figure}[b]
    \centering
    \includegraphics[width=0.7\textwidth]{graphs/idf-scores}
    \caption{IDF scores for concepts in the annotations}
    \label{fig:idf-scores}
\end{figure}

\begin{figure}[b]
    \centering
    \includegraphics[width=0.7\textwidth]{graphs/tf-scores}
    \caption{Term Frequency scores for concepts in the annotations}
    \label{fig:tf-scores}
\end{figure}

Textual annotations accounted for the highest amount of time taken on average in the collection process. The largest number of annotations collected for a methodology are the query annotations. Qualitative feedback from annotators note that the relevance assessments are the most tedious to collect. Intuitively, formulating a query for a typical search engine does not take a very long time, which can account for the marginal average time for this annotation type. On the other hand, composing (in the annotators mind and physically typing on a keyboard) a descriptive paragraph filled with context and semantics leads to the conclusion that textual annotations do, in fact, take a significant amount of time to annotate. In the same manner, the process of completing a relevance assessment involves scrolling and clicking a multitude of times; this time adds up and is evident in the reported average time.

\FloatBarrier
\section{Retrieval Effectiveness}

Results of each experiment are reported as a table which provides the results from \verb|trec_eval| and as a precision-recall graph. The results from only the manually annotated images are displayed first. Table \ref{table:manual-results} presents the \verb|trec_eval| results for the four annotation methodologies \textit{and} the result of combining all four of the methodologies. The  NTCIR-12 LSAT topics are used for experiments. Among other fields, each topic contains a title and a description; these fields are read as input queries. The title field is more representative of what a typical query looks like when issued by a user. The results of running these experiments are visualised as precision-recall graphs in Figures \ref{fig:manual-result-title} and \ref{fig:manual-result-desc}.

In experiments that use the title field as an input the query annotations perform the best of the four methodologies, however when combining all four collections of annotations together the effectiveness increases slightly, and the precision is higher than the query annotations at high recall (more relevant results). Combining annotations for the experiments that use the description field outperform all four of the methodologies, and the query annotations perform significantly worse.

\begin{table}[ht]
    \begin{tabular}{|c|c|c|c|c|}
        \multicolumn{5}{c}{Topics Titles}\\ \hline
         Methodology & MAP & RR & P@10 & Relevant Retrieved \\ \hline
         Text & 0.3442$^{qc}$ & 0.9223$^{c}$ & 0.8333$^{rc}$ & 1062 \\ \hline
         Tag & 0.5468$^{qcd}$ & 0.9578$^{d}$ & 0.8396$^{rcd}$ & 1040 \\ \hline
         Query & 0.6400$^{tgad}$ & 0.9653$^{rd}$ & 0.8750$^{rd}$ & 1559 \\ \hline
         Relevance Assessment & 0.4815$^{qc}$ & 0.8406$^{qc}$ & 0.6917$^{tgqrd}$ & 831 \\ \hline
         Combined & 0.6495$^{tgrc}$ & 1.000$^{ta}$ & 0.9062$^{tgr}$ & 1612 \\ \hline
         \multicolumn{5}{c}{Topic Descriptions} \\ \hline
         Methodology & MAP & RR & P@10 & Relevant Retrieved \\ \hline
         Text & 0.5285$^{gqc}$ & 0.9792$^{gq}$ & 0.8521$^{gqc}$ & 1088 \\ \hline
         Tag & 0.4627$^{trcd}$ & 0.8928$^{tqcd}$ & 0.7687$^{tqcd}$ & 1055 \\ \hline
         Query & 0.4457$^{tcd}$ & 0.7683$^{tgrcd}$ & 0.5958$^{tgrcd}$ & 1566 \\ \hline
         Relevance Assessment & 0.5219$^{c}$ & 0.9271$^{q}$ & 0.7938$^{tqcd}$ & 972 \\ \hline
         Combined & 0.5855$^{tgqrc}$ & 0.9815$^{gq}$ & 0.8875$^{tgqr}$ & 1613 \\ \hline         
    \end{tabular}
    \caption{MAP, Reciprocal Rank, Precision at 10 scores, and number of relevant retrieved images for the manual annotations. Two tails t-test statistical significance ($p<0.05$) is indicated through the following labels for inter-measurement: text $^t$, tags $^g$, query $^q$, relevance assessment $^r$, combined $^c$. Statistical significance between the same methodology is represented as $^d$.}
    \label{table:manual-results}
\end{table}

\begin{figure}[ht]
    \centering
    \includegraphics[width=0.8\textwidth]{graphs/manual-title}
    \caption{Precision-recall curves for the manual annotations using topic titles}
    \label{fig:manual-result-title}
\end{figure}

\begin{figure}[ht]
    \centering
    \includegraphics[width=0.8\textwidth]{graphs/manual-desc}
    \caption{Precision-recall curves for the manual annotations using topic descriptions}
    \label{fig:manual-result-desc}
\end{figure}

\FloatBarrier

A neural network framework (neuraltalk2) is trained on the manual textual, tag, and query annotations (i.e. after the manual annotations are collected). Neuraltalk2 is able to produce captions for every image in the collection. The quality of these captions is summarised in Table \ref{table:learnt-results} -- an unfortunate result which could be attributed to the amount of training data. The automatic captions that are generated are evaluated in the same way as the manual annotations. The output of the neural network architecture is formatted to be used in the evaluation pipeline as described in Section \ref{methods:evaluating}. 

There is no clear individual annotation methodology that outperforms the others, the scores are too low to indicate this. Combining the three automatic annotations together does seem to increase the overall precision. The learnt queries do retrieve the most number of images, in a similar result to the manual annotation results.

The number of iterations the neural network architecture covered for each annotation methodology is visualised in Figures \ref{fig:val-loss-1} and \ref{fig:val-loss-2}. The results of the automatic captioning do not get better over time -- in fact they get worse. The gap between topics with a large number of relevant images, topics with a low number of images, and the low number of training examples is presumably the contributing factor to the results in these figures.

\begin{table}[ht]
    \centering
    \begin{tabular}{|c|c|c|c|c|}
        \multicolumn{5}{c}{Topic Titles}\\ \hline
         Methodology & MAP & RR & P@10 & Relevant Retrieved\\ \hline
         Text & 0.0048 & 0.0248 & 0.0196 & 187 \\ \hline
         Tag & 0.0083$^{c}$ & 0.0184 & 0.0022 & 136 \\ \hline
         Query & 0.0076 & 0.0174 & 0.0021 & 246 \\ \hline
         Combined & 0.0164$^{g}$ & 0.0393 & 0.0169 & 337 \\ \hline
        \multicolumn{5}{c}{Topic Descriptions}\\ \hline
         Methodology & MAP & RR & P@10 & Relevant Retrieved\\ \hline
         Text & 0.0048 & 0.0232 & 0.0167 & 222 \\ \hline
         Tag & 0.0050 & 0.0184 & 0.0063 & 145 \\ \hline
         Query & 0.0051 & 0.0062 & 0.0062 & 199 \\ \hline
         Combined & 0.0096 & 0.0470 & 0.0187 & 325 \\ \hline
    \end{tabular}
    \caption{MAP, Reciprocal Rank, Precision at 10 scores, and number of relevant retrieved images for the automatically generated annotations. Two tails t-test statistical significance ($p<0.05$) is indicated through the following labels for inter-measurement: text $^t$, tags $^g$, query $^q$, relevance assessment $^r$, combined $^c$. Statistical significance between the same methodology is represented as $^d$.}
    \label{table:learnt-results}
\end{table}

\begin{figure}[ht]
    \centering
    \includegraphics[width=0.8\textwidth]{graphs/auto-title}
    \caption{Precision-recall curves for the learnt annotations using titles}
    \label{fig:auto-result-title}
\end{figure}

\begin{figure}[ht]
    \centering
    \includegraphics[width=0.8\textwidth]{graphs/auto-desc}
    \caption{Precision-recall curves for the learnt annotations using descriptions}
    \label{fig:auto-result-desc}
\end{figure}

\begin{figure}
    \centering
    \includegraphics[width=0.7\textwidth]{graphs/initial-validation-loss-history}
    \caption{Validation loss history (\textless 100,000 iterations)}
    \label{fig:val-loss-1}
\end{figure}

\begin{figure}
    \centering
    \includegraphics[width=0.7\textwidth]{graphs/validation-loss-history}
    \caption{Validation loss history (\textgreater 200,000 iterations)}
    \label{fig:val-loss-2}
\end{figure}
\chapter{Discussion}

This chapter discusses the effectiveness of the annotation methodologies, the automatic image annotation results, and the annotation interfaces. Each section aims to explain and summarise the results and evaluate the significance of the results.

\section{Annotation Effectiveness}

The results from these experiments outlined in Chapter \ref{chapter:results} indicate that of all the types of annotation methodologies, the query annotations provide the best performance at the lowest cost (i.e. the time it took to annotate). Utilising multiple annotations can increase the interpolated precision further towards total recall. In fact at total recall combining all of the annotations achieves the highest precision overall. It is not surprising to see the tag annotations performing badly in both experiments, since these annotations are unable to encode semantic meaning like text and queries. For instance, consider images that fall into the topic `Building a Computer': the tag `building' can have more than one meaning, a physical structure and the act of assembling an object. The tag annotations did poorly in this topic, whereas the textual annotations did the best -- these contain semantics which the search engine can exploit when ranking images.  What is surprising (at least when using the description as an input query), is that for some topics assessment annotations perform better. This is most likely due to the fact that the weights of annotated concepts allow the search engine to rank these relevant documents higher. The performance of the textual annotations can perhaps be attributed to spelling and grammatical errors; there is difficulty associated with ensuring these errors are corrected during a pre-processing step in addition to blocking incorrect annotations from being entered in the first place.

In retrospective, the concepts for each query should be chosen manually based on the topics, but this would render a retrieval system unusable behind topics for which relevant concept annotations have not been defined a priori. When choosing concepts for use in relevance assessment, there is a large overlap between the titles and the descriptions of the topics and this was seen as good enough. Prior to analysing the results of the relevance assessment annotations, it was thought that they might perform only as well as the images retrieved using tag annotations. Similar to tags, the concepts of the relevance assessments do not contain semantic meaning; however unlike tags, each concept is assigned a weighting of how relevant it is to an image. In ascribing weights, the concepts assigned to images are contextualised; now `building' is highly relevant \textit{and} `computer' is highly relevant, as opposed to another image that may have an equally high weight for `building' but a high weight for `architecture'. The image with the high `computer' concept will be ranked higher because of the weighting; explaining why relevance assessments outperform the tags even though less images are retrieved. The trade-off in the end is that tagging takes half as long for slightly worse retrieval performance. An assumption that can be made for the relevance assessment annotations then is that if more concepts are added that are relevant to the topics, the retrieval effectiveness can be achieved.

In the case of the query annotations performing the best out of the four annotation methodologies, one would assume that it is simply due to the fact that more images are annotated. The results of the relevance assessments indicate that this is not the case -- the effectiveness of the query methodology can not solely be attributed to the number of images annotated. The relevance assessment annotations perform better (where the input query is taken from the description of the topic) than the query, tags, and in some cases the textual annotations, having the least number of images retrieved. It would seem then that the query annotations are highly suited to the experiments where the input query is composed of very few terms (much like a typical query). It is also possible that, being significantly shorter than the textual annotations, there is less error for spelling and grammatical mistakes. Moreover, the low cost of annotation time further makes the query annotation methodology the most attractive option for annotating a collection of lifelog images for use in text-based image retrieval.

\section{Automatic Image Annotation}

In total, there are 6,657 images that the NTCIR-12 LSAT organisers consider relevant from a collection of 88,125 images (only 7.5\%). If images are annotated using the clustering method of sampling images, only 16,014 images (18\% of the data set) could be annotated. Of that, 1,176 images are considered relevant. Sampling known relevant images is highly important -- the clusters do not provide a good enough distribution of relevant and non-relevant images to annotate if we just pick at random. A distribution that consists of more relevant images than non-relevant images (i.e. relevant to topics) would appear to be more appropriate if one is interested in using annotations for training data. Intuitively it would also seem that topics with a high number of relevant images perform badly due to a low number of training examples; and topics with a low number of relevant images achieve above $0$ precision at high recall because these topics have all of their images manually annotated (remember that a \textit{maximum} of $30$ images are chosen from the known relevant images). This intuition is made visible in Figure \ref{fig:learnt-queries}. The query annotations are used as a demonstration as they are the most interesting of all the learnt annotation methodologies in that they are able to retrieve the highest number of images.


Next, the results obtained when automatically annotating images using the neural network architecture examined in this thesis are discussed. The selection of image to be annotated also intuitively impacts the way a neural network captions images. If a topic contains a diverse range of locations and actions but the images sampled for annotation only cover one of these locations then it becomes obvious that many images will get mis-captioned. In hindsight, taking the time to select a variety of perspectives and locations within a moment to cover edge cases could increase the accuracy of the captions; but it is unknown by how much. Another factor that contributes to the poor retrieval effectiveness is the number of training examples required --- there simply may not be enough annotations for the neural network to learn. One thing that can be said for certain is that the number of training iterations required is not the limiting factor in generating accurate captions. Over 200,000 iterations were performed for the three methodologies, and all of them got worse over time. An acceptable number of iterations for this particular setup appears to fall between 30,000 and 100,000; depending on the type of annotation.

\FloatBarrier
\begin{figure}[h]
    \centering
    \includegraphics[width=0.8\textwidth]{graphs/learnt-queries}
    \caption{Breakdown of running the same experiment on topics with a large proportion of relevant images (\textgreater 300), the middle number of relevant images (between 100 and 300), and topics with a low number of relevant images (\textless 100)}
    \label{fig:learnt-queries}
\end{figure}

\begin{figure}[h]
    \centering
    \includegraphics[width=0.8\textwidth]{graphs/relevant-images}
    \caption{Relevant images in each topic}
    \label{fig:relevant-images}
\end{figure}
\FloatBarrier

The size of the collection and the number of manual annotations are the most likely reasons for the poor performance when automatically captioning images. Compare for example the number of annotations collected as part of these experiments with the MSCOCO data set~\cite{lin2014microsoft} which contains more than 300,000 images with five annotations per image. The number of relevant images in each topic may also be a problem: many topics have a less than 200 relevant images associated with them (around 2\% of the actual collection). The number of images in each topic is detailed in Figure \ref{fig:relevant-images}

\section{Interfaces}

The original intent of the topics perhaps does not appear to suite annotating individual images sampled at random. Relevant images in a topic are actually a group of images, or a `moment'. Each topic has several relevant moments, which contain certain images that do not appear to be relevant to the topic at all. As a general rule the images that do not look relevant, even though they were considered to be so by the task organisers were are ignored. For instance, in the topic `Conversation while eating', the lifelogger often wipes his mouth with a serviette blocking the camera. These blurry or obscured images are not annotated. 

\begin{figure}[b]
    \centering
    \includegraphics[width=0.8\textwidth]{images/new-rel-ass-interface}
    \caption{Updated query annotation interface}
    \label{fig:new-rel-ass}
\end{figure}

\FloatBarrier
Grouping images into moments may also speed up the annotation process: annotating several images considered to be a moment may not only allow collecting annotations less tedious and time consuming, but could also allow for more images to be annotated. The downside to this, however, may be that these irrelevant images crop up inside each moment. One way to avoid a situation like this is to let the annotator remove images that are covered by objects or too blurry to make anything out.

The interfaces themselves are iterative and dynamic while collecting annotations; they change often in response to feedback from annotators. The biggest change from the initial design is the relevance assessment interface, as seen in Figure \ref{fig:new-rel-ass}. Now rather than clicking to judge every concept to the image, each concept is grouped alphabetically; when one is clicked, assessments are added underneath the image. Not every caption must be assessed manually: When the next button is clicked, all unassessed captions are automatically considered not relevant. 




% - While there is an overlap of the terms in the annotations and queries, this does not necessarily mean that the terms in the annotations are used within the same context. This is particularly evident in the tag annotations. The term `key' is relevant to a query where the person wearing a lifelog camera is getting a key replaced, but the term was also found within the context of someone using a `key card' to enter their office.

% The topics were not designed to have images annotated individually out of order. Topics were designed with a range of images that are relevant, or a `moment' of images. 

% - even though images are chosen from random from a smaller pool of images, it simply was not enough to pick at random. The qrels identified 6657 images that should be relevant. From here, the pool of sampled images only contains 1179 images. Only 17.7\% of the images in the images chosen by clustering can be annotated.


% differences between each annotation methodology - All of them are very different to each other, which presents some interesting problems such as how long it takes to annotate each image, how can each annotation methodology be evaluated etc

% text
% Textual annotation are the most time consuming annotation type to collect. Many images contain two or more highly relevant events or `important' objects. Describing these can take annotators up to several minutes each. This is opposed to all the other annotation methodologies where most images can be annotated in a minute or less.

% tags
% In terms of evaluation, tags may be the opposite of textual annotations; intuitively tags should be the best at training an image classifier and are expected to perform the worst when embedded a search task. Tags, however, may be very good at boosting the performance of other annotation types in the search task when combined. For instance, searching on the text \textit{and} tags fields may increase the scores of text annotation alone.

% query
% and therefore it is unknown how well the collected queries will perform in search or for training an image classifier. The level of detail in these annotations will be very low, since most queries should be short in length.
% Much like tags, this annotation type will be very easy to collect. Formulating a query for an image should not takz a significant amount of time. There is the problem of bias \todo{Is there really? I should investigate this further}
\chapter{Conclusions}

\section{Contribution Summaries}

\textbf{How can annotations for a collection of lifelog images be evaluated?}

A gold standard annotation does not exist for these lifelog images: individual comparison of annotations is not possible. An ad-hoc TREC-style is instead the technique for evaluating annotations. This extrinsic evaluation method is also the method chosen for the NTCIR-12 lifelog semantic access task. The focus of this research is to investigate the best methodology for annotating images for use in a text-based information retrieval system. The other meta data associated with lifelogs is ignored, in preference of focusing the attention on the images. Knowledge of the retrieval effectiveness of each annotation methodology within the context of a text-based image retrieval system allows the research to concentrate on tuning retrieval models, automatically captioning images, and incorporating the aforementioned meta data. 

\textbf{What are the possible ways lifelog images can be annotated and what is their retrieval effectiveness?
}

Four annotation methodologies are chosen for determining their retrieval effectiveness. Of the selected types of annotations: text, tags, queries, and relevance assessment -- the query annotation is the most effective in a text-based image retrieval search task. In fact, not only is the query annotation methodology the most effective, it is the cheapest to collect in that the average time to collect one type of this annotation is the lowest of all four methodologies. The second most effective annotation methodology for lifelog images are relevance assessments, however these are time consuming to collect and require prior selection of concepts to judge. The textual and tag annotations have about the same retrieval effectiveness; the difference is that the textual annotations take double the time on average to annotate. The retrieval effectiveness of the annotations is slightly improved when combining all four together, particularly at high recall.

\textbf{How do annotations generated by current state-of-the-art automatic image captioning techniques compare to the effectiveness of manual annotations?}

The effectiveness of annotations generated by one particular state-of-the-art automatic captioning technique is not comparable to the effectiveness of manual annotations. The expectation of providing more training data in combination with some tuning is that the effectiveness will improve. Current results show, however, that the query annotations still retrieve the most number of images overall. Again, combining the annotations increases the overall effectiveness, especially at low recall.

\section{Future Work}

\textbf{Moment Annotation}

The most important change to make if continuing this research is to segment the collection of images into moments rather than individual images. This would allow many more images to be annotated in a smaller amount of time and cover more edge cases. New techniques are required to sample images into moments rather than individually, and the image annotation interfaces must be updated to reflect this. 

Adding and removing images from each moment in the annotation interfaces is necessary (the sampling process is not likely to perfectly capture all relevant images in a moment i.e there could be outliers on the edges of the moment). One thing that needs further investigation is whether to overlap moments with each other. Is it sensible to annotate images more than once if they appear in moments which overlap?

\textbf{Automatic Captioning}

Automatic captioning tools seem to work very well in other domains and it is unfortunate that this research could not exploit them. The number of annotations used as training data is the most likely reason this research could not generate captions effectively. Although diverse in moments and objects, there are still less than 100,000 images in total, with less than 20,000 of these images considered `relevant'. Many topics have less than 100 relevant images so using a data set with more relevant images may improve how images get captioned. 

The retrieval system itself can also be expanded up to combine TBIR and CBIR systems. There has been recent work done to generate images from textual descriptions by Reed et. al~\cite{reed2016generative}; however it looks limited to small categories and the resolution of the generated images are low. It is not difficult to imagine a system which generates images from a query and retrieves using an image similarity technique. As of writing this, the idea is only a pipe dream; the future of automatic captioning and image retrieval of lifelogs is bright.

\textbf{Image sampling}

Given more time, it would be worthwhile investigating other image similarity measures (such as those used by the LEMoRe team ~\cite{de40lemore} at NTCIR-12) to possibly produce a more uniformly distributed sample set.

% There has been some work on generating images based on textual descriptions, but it looks limited to small collections of images, and does not seem to be able to produce large resolution images. In the future, search may be able to be performed by image description->image generation->image matching \todo{this https://github.com/paarthneekhara/text-to-image and the paper should be cited here}
 
% Current state of the art automatic image captioning algorithms~\cite{karpathy2015deep} allow many images to be textually annotated with some level of accuracy. 
 
% The collection of images used is still relatively small. Although diverse in moments and objects, there are still less than 100k images. A large-scale test collection in the magnitude of hundreds of thousands of images to a million or more images could offer better results. \todo{I have read some discussions on line about this but I am yet to find any papers/formal research on this, as well as the exact number of images}. Results will only get more accurate because there is a more diverse range of events happening in the collection.

% There are four annotation types under investigation. Each one is very different to the last. There have been no studies into the most effective methodology or set of methodologies for annotating lifelog images. It would be nice to investigate more than four, however due to time constraints this is inconceivable.

% \begin{appendices}
% \renewcommand\thetable{\thesection\arabic{table}}
% \renewcommand\thefigure{\thesection\arabic{figure}}
% \section{Format of Collection} \label{app:format}
% \end{appendices}

\bibliographystyle{abbrv}
\bibliography{thesis}

\end{document}
